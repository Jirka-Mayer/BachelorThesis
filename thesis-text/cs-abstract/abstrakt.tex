%%% Šablona pro jednoduchý soubor formátu PDF/A, jako treba samostatný abstrakt práce.

\documentclass[12pt]{report}

\usepackage[a4paper, hmargin=1in, vmargin=1in]{geometry}
\usepackage[a-2u]{pdfx}
%\usepackage[czech]{babel}
\usepackage[utf8]{inputenc}
\usepackage[T1]{fontenc}
\usepackage{lmodern}
\usepackage{textcomp}

\begin{document}

%% Nezapomeňte upravit abstrakt.xmpdata.

Automatický přepis notových zápisů (Optical Music Recognition) je zajímavá úloha, v mnoha ohledech podobná automatickému přepisu textu (Optical Character Recognition). Přináší s sebou ovšem mnoho problémů, které způsobují potíže klasickým metodám počítačového vidění. Hluboké neuronové sítě umožnily řešit automatický přepis textu tzv. end-to-end přístupem, kdy se celá úloha řeší najednou. Zkusili jsme použít tuto metodu na problém rozpoznávání notových zápisů, ale zaměřili jsme se pouze na ručně psané zápisy. Pro vyřešení nedostatku trénovacích dat jsme vyvinuli sázecí systém s názvem Mashcima. Tento systém úspěšně napodobuje vzhled datasetu CVC-MUSCIMA. Provedli jsme vyhodnocení našeho přístupu na části datasetu CVC-MUSCIMA s velmi nadějnými výsledky, což naznačuje, že použité řešení je funkční a další práce v tomto směru by mohla vést ještě k dalšímu zlepšení.

\end{document}
